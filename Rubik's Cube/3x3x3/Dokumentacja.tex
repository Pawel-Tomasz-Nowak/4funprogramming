\documentclass{article}
\usepackage[utf8]{inputenc}
\usepackage{polski}
\usepackage[left=5px,top=5px]{geometry}
\usepackage{xcolor}
\usepackage{amsmath}
\usepackage{amssymb}
\usepackage{amsthm}



\begin{document}
Zacznijmy od zaprojektowania  klasy reprezentującej jedną ściankę kostki 3x3x3. Wiemy, że składa sie ona z 3x3 = 9 fragmentów: czterech dwukolorowych klocków, czterech trzyelementowych klocków oraz jednego, centralnego, jednokolorowego klocka. 
Centralny klocek określa kolor (kolor pełni funkcję atrybutu rozróżniającego) ścianki.
Dla konwencji przyjmijmy, następujące kodowanie kolorów:

\begin{description}
\item[Czerwony] \textcolor{red}{1}
\item[Pomarańczowy] \textcolor{red}{2}
\item[Zielony] \textcolor{green}{3}
\item[Niebieski] \textcolor{green}{4}
\item[Biały] \textcolor{blue}{5}
\item[Żółty] \textcolor{blue}{6}.
\end{description}
Pary (Czerwony, pomarańczowy), (Zielony, Niebieski),(Biały,Żółty) są naturalnymi parami.
Można znaleźć fajny wzór, który pozwoli nam znaleźć partnera kolorowego dla danego koloru. Otóż dla numeru n partnerem jest:

\begin{equation}
Partner(n) = \begin{cases}
n+1, \text{gdy n nieparzyste} \\
n-1, \text{gdy n parzyste}

\end{cases}
\end{equation}


Położenie ścianki można opisać za pomocą sześciu liter: F (Front), B (Back), L (Left), R (Right),D (Down), U (Up), które zakodujemy podobnie:
\begin{description}
\item[F] \textcolor{red}{1}
\item[B] \textcolor{red}{2}
\item[L] \textcolor{green}{3}
\item[R] \textcolor{green}{4}
\item[D] \textcolor{blue}{5}
\item[U] \textcolor{blue}{6}. \\
\end{description}

\end{document}