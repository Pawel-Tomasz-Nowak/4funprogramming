\documentclass{article}
\usepackage[utf8]{inputenc}
\usepackage{polski}
\usepackage[left=5px,top=5px]{geometry}
\usepackage{xcolor}
\usepackage{amsmath}
\usepackage{amssymb}
\usepackage{amsthm}



\begin{document}
\section{Dozwolone ruchy}
Zacznijmy od zdefiniowana każdego możliwego ruchu.

\subsection{Obrót kostki}
\begin{enumerate}
\item[1] z obrót kostki w lewo, tak aby środek ściany twarzowej poruszał się po osi z. (Niedolna i niegórna ściana nie zmieniają pozycji).

\item[2] z obrót kostki w prawo, tak aby środek ściany twarzowej poruszał się po osi z. (Niedolna i niegórna ściana nie zmieniają pozycji)
\item[3] x - obrót całej kostki "z dala od ciebie" (nielewa i nieprawa ściana nie zmieniają pozycji)
\item[4] x' obrót kostki "w twoją stronę" (nielewa i nieprawa ściana nie zmieniają pozycji)
\item[5] y' obrót kostki w lewo, tak aby środek ściany twarzowej poruszał się po osi y.
\item[5] y' obrót kostki w prawo, tak aby środek ściany twarzowej poruszał się po osi y.

\end{enumerate}

\subsection{Obrót ścianek}
ZR - Zgodnie z ruchem wskazówek zegara
PR - Przeciwnie do ruchu wskazówek zegara.

\begin{enumerate}
\item[1] F -  obrót przedniej ściany ZR.
\item[2] F' - obrót przedniej ściany PR 
\item[3] B-  obrót tylnej ściany ZR.
\item[4] B' - obrót tylnej ściany PR  
\item[5] - L  obrót lewej ściany ZR.
\item[6] -L' obrót lewej ściany PR 
\item[7] - P  obrót prawej ściany ZR.
\item[8] -P' obrót prawej ściany PR 
\item[9] - D  obrót dolnej ściany ZR.
\item[10] -D' obrót dolnej ściany PR 
\item[11] - U  obrót górnej ściany ZR.
\item[12] - U' obrót górnej ściany PR 

\end{enumerate}


\section{Klasy}
Zacznijmy od zaprojektowania  klasy reprezentującej jedną ściankę kostki 3x3x3. Wiemy, że składa sie ona z 3x3 = 9 fragmentów: czterech dwukolorowych klocków, czterech trzyelementowych klocków oraz jednego, centralnego, jednokolorowego klocka. 
Centralny klocek określa kolor (kolor pełni funkcję atrybutu rozróżniającego) ścianki.
Dla konwencji przyjmijmy, następujące kodowanie kolorów:

\begin{description}
\item[Czerwony] \textcolor{red}{1}
\item[Pomarańczowy] \textcolor{red}{2}
\item[Zielony] \textcolor{green}{3}
\item[Niebieski] \textcolor{green}{4}
\item[Biały] \textcolor{blue}{5}
\item[Żółty] \textcolor{blue}{6}.
\end{description}
Pary (Czerwony, pomarańczowy), (Zielony, Niebieski),(Biały,Żółty) są naturalnymi parami.
Można znaleźć fajny wzór, który pozwoli nam znaleźć partnera kolorowego dla danego koloru. Otóż dla numeru n partnerem jest:

\begin{equation}
Partner(n) = \begin{cases}
n+1, \text{gdy n nieparzyste} \\
n-1, \text{gdy n parzyste}

\end{cases}
\end{equation}


Położenie ścianki można opisać za pomocą sześciu liter: F (Front), B (Back), L (Left), R (Right),D (Down), U (Up), które zakodujemy podobnie:
\begin{description}
\item[F] \textcolor{red}{1}
\item[B] \textcolor{red}{2}
\item[L] \textcolor{green}{3}
\item[R] \textcolor{green}{4}
\item[D] \textcolor{blue}{5}
\item[U] \textcolor{blue}{6}. \\
\end{description}

\end{document}